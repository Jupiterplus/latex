\section{Etcd}

\begin{frame}
    \framtitle{etcd}
etcd = /etc distributed
\begin{itemize}
    \item Key-Value Storage
    \item Consistency
    \item High Availability
    \item Failure Tolerant
    \item Cluster Configuration
\end{itemize}
\end{frame}


\begin{frame}
    \framtitle{etcd}
\begin{itemize}
    \item Open source developed by CoreOS
    \item Written in Go
    \item Durable
    \item Watchable
    \item Exposed via HTTP
    \item Runtime Recofigurable
\end{itemize}
\end{frame}

\begin{frame}
    \frametitle{Etcd}

Service registration relies on using a key TTL along with heartbeating from the service to ensure the key remains available. If a services fails to update the key's TTL, etcd will expire it. If a service becomes unavailable, clients will need to handle the connection failure and try another service instance.

\end{frame}

\begin{frame}
    \frametitle{Etcd Pros and Cons}

\begin{itemize}
    \item Pros
        \begin{itemize}
            \item Service discovery involves listing the keys under a directory and then waiting for changes on the directory. Since the API is HTTP based, the client application keeps a long-polling connection open with the Etcd cluster.
            \item Has been around for longer than Consul. 150\% more watched/stars.
            \item 3 times as many contributors(i.e. more eyes) and forks on github.
        \end{itemize}
    \item Cons
\end{itemize}
\end{frame}
