\begin{frame}
    \frametitle{Q \& A}
    \begin{block}{Question}
I don't understand why the blockchain is so important. Isn't the
requirement for the owner's signature on each transaction enough to
prevent bitcoins from being stolen?
    \end{block}
\end{frame}

\begin{frame}
    \frametitle{Q \& A}
    \begin{block}{Question}
I don't understand why the blockchain is so important. Isn't the
requirement for the owner's signature on each transaction enough to
prevent bitcoins from being stolen?
    \end{block}

    \begin{block}{Answer}
The signature is not enough, because it doesn't prevent the owner
from spending money twice: signing two transactions that transfer the
same bitcoin to different recipients. The blockchain acts as a
publishing system to try to ensure that once a bitcoin has been spent
once, lots of participants will know, and will be able to reject a
second spend.
    \end{block}
\end{frame}

\begin{frame}
    \frametitle{Q \& A}
    \begin{block}{Question}
Why does Bitcoin need to define a new currency? Wouldn't it be more
convenient to use an existing currency like dollars?
    \end{block}
\end{frame}

\begin{frame}
    \frametitle{Q \& A}
    \begin{block}{Question}
Why does Bitcoin need to define a new currency? Wouldn't it be more
convenient to use an existing currency like dollars?
    \end{block}

    \begin{block}{Answer}
The new currency (Bitcoins) allows the system to reward miners with
freshly created money; this would be harder with dollars because it's
illegal for ordinary people to create fresh dollars. And using dollars
would require a separate settlement system: if I used the blockchain
to record a payment to someone, I still need to send the recipient
dollars via a bank transfer or physical cash.
    \end{block}
\end{frame}

\begin{frame}
    \frametitle{Q \& A}
    \begin{block}{Question}
Why is the purpose of proof-of-work?
    \end{block}
\end{frame}

\begin{frame}
    \frametitle{Q \& A}
    \begin{block}{Question}
Why is the purpose of proof-of-work?
    \end{block}

    \begin{block}{Answer}
It makes it hard for an attacker to convince the system to switch
to a blockchain fork in which a coin is spent in a different way than
in the main fork. You can view proof-of-work as making a random choice
over the participating CPUs of who gets to choose which fork to
extend. If the attacker controls only a few CPUs, the attacker won't
be able to work hard enough to extend a new malicious fork fast enough
to overtake the main blockchain.
    \end{block}
\end{frame}

\begin{frame}
    \frametitle{Q \& A}
    \begin{block}{Question}
Could a Bitcoin-like system use something less wasteful than
proof-of-work?
    \end{block}
\end{frame}

\begin{frame}
    \frametitle{Q \& A}
    \begin{block}{Question}
Could a Bitcoin-like system use something less wasteful than
proof-of-work?
    \end{block}

    \begin{block}{Answer}
Proof-of-work is hard to fake or simulate, a nice property in a
totally open system like Bitcoin where you cannot trust anyone to
follow rules. There are some alternate schemes; search the web for
proof-of-stake, for example. In a smallish closed system, in which the
participants are known though not entirely trusted, Byzantine
agreement protocols could be used, as in Hyperledger.
    \end{block}
\end{frame}

\begin{frame}
    \frametitle{Q \& A}
    \begin{block}{Question}
Can Alice spend the same coin twice by sending "pay Bob" and "pay
Charlie" to different subsets of miners?
    \end{block}
\end{frame}

\begin{frame}
    \frametitle{Q \& A}
    \begin{block}{Question}
Can Alice spend the same coin twice by sending "pay Bob" and "pay
Charlie" to different subsets of miners?
    \end{block}

    \begin{block}{Answer}
The most likely scenario is that one subset of miners finds the
nonce for a new block first. Let's assume the first block to be found
is B50 and it contains "pay Bob". This block will be flooded to all
miners, so the miners working on "pay Charlie" will switch to mining a
successor block to B50. These miners validate transactions they place
in blocks, so they will notice that the "pay Charlie" coin was spent in
B50, and they will ignore the "pay Charlie" transaction. Thus, in this
scenario, double-spend won't work.

There's a small chance that two miners find blocks at the same time,
perhaps B50' containing "pay Bob" and B50'' containing "pay Charlie". At
this point there's a fork in the block chain. These two blocks will be
flooded to all the nodes. Each node will start mining a successor to one
of them (the first it hears). Again the most likely outcome is that a
single miner will finish significantly before any other miner, and flood
the successor, and most peers will switch that winning fork. The chance
of repeatedly having two miners simultaneously find blocks gets very
small as the forks get longer. So eventually all the peers will switch
to the same fork, and in fork there will be only one spend of the coin.

The possibility of accidentally having a short-lived fork is the reason
that careful clients will wait until there are a few successor blocks
before believing a transaction.
    \end{block}
\end{frame}

\begin{frame}
Q: It takes an average of 10 minutes for a Bitcoin block to be
validated. Does this mean that the parties involved aren't sure if the
transaction really happened until 10 minutes later?
 
A: The 10 minutes is awkward. But it's not always a problem. For
example, suppose you buy a toaster oven with Bitcoin from a web site.
The web site can check that the transaction is known by a few servers,
though not yet in a block, and show you a "purchase completed" page.
Before shipping it to you, they should check that the transaction is
in a block. For low-value in-person transactions, such as buying a cup
of coffee, it's probably enough for the seller to ask a few peers to
check that the bitcoins haven't already been spent (i.e. it's
reasonably safe to not bother waiting for the transaction to appear in
the blockchain at all).
\end{frame}

\begin{frame}
Q: What can be done to speed up transactions on the blockchain?
 
A: I think the constraint here is that 10 minutes needs to be much
larger (i.e. >= 10x) than the time to broadcast a newly found block to
all peers. The point of that is to minimize the chances of two peers
finding new blocks at about the same time, before hearing about the
other peer's block. Two new blocks at the same time is a fork; forks
are bad since they cause disagreement about which transactions are
real, and they waste miners' time. Since blocks can be pretty big (up
to a megabyte), and peers could have slow Internet links, and the
diameter of the peer network might be large, it could easily take a
minute to flood a new block. If one could reduce the flooding time,
then the 10 minutes could also be reduced.
\end{frame}

\begin{frame}
Q: The entire blockchain needs to be downloaded before a node can
participate in the network. Won't that take an impractically long time
as the blockchain grows?

A: It's true that it takes a while for a new node to get all the
transactions. But once a given server has done this work, it can save
the block chain, and doesn't need to fetch it again. It only needs to
know about new blocks, which is not a huge burden. I think most
ordinary users of Bitcoin won't run full Bitcoin nodes; instead they
will one way or another trust some full nodes.
\end{frame}

\begin{frame}
Q: Is it feasible for an attacker to gain a majority of the computing
power among peers?

A: It may be feasible; some people think that big cooperative groups
of miners have been close to a majority at times:
http://www.coindesk.com/51-attacks-real-threat-bitcoin/
\end{frame}


